\documentclass[times, utf8, seminar, numeric]{fer}
\usepackage{booktabs}
 \usepackage{url}

\begin{document}

% Ukljuci literaturu u seminar
\nocite{*}

% TODO: Navedite naslov rada.
\title{Vlastita implementacija filtera: \protect\\ The Logarithmic Dynamic Cuckoo Filter}

% TODO: Navedite vaše ime i prezime.
\author{Fran Ostroški, Elena Wachtler}

% TODO: Navedite ime i prezime voditelja.
\voditelj{Mirjana Domazet-Lošo}

\maketitle

\tableofcontents

\chapter{Uvod}
Spomenuti glavnu temu projekta, da se radi o projektu u sklopu kolegija Bioinformatika 1. Spomenuti sve izvore iz literature, točno što je koji od njih napravio i kako se jedan razlikuje od drugog (ovo se boduje, čak 3 boda, baš se moraju spomenuti izvori u tekstu, tako piše u uputama).

NAPOMENA: u bibtex fileu nedostaje jedan izvor, nisam mogla naći njegov .bib: 
Fan et al. 2013. Cuckoo Filter: Better Than Bloom; https://www.cs.cmu.edu/~binfan/papers/login cuckoofilter.pdf - našla sam samo za Practically Better Than Bloom.

\chapter{Opis algoritma}
Mogu se i ovdje koristiti izvori, možda čak i bolje nego da se koriste u Uvodu, iako se mogu i u uvodu spomenuti.

Vizualizacija algoritma na jednostavnom primjeru.

\chapter{Analiza}
Analiza točnosti, vremena izvođenja i utroška memorije za različite testne slučajeve - ovo nosi 3 boda. Testira se na stvarnim podatcima, kod nas E. coli - rezultati prikazani u tablici ili grafu.

\chapter{Zaključak}
Zaključak.

\bibliography{literatura}
\bibliographystyle{fer}

\chapter{Sažetak}
U ovome radu dan je pregled vlastite implementacije rješenja problema opisanog u znanstvenim radovima koja je napravljena u sklopu projekta iz kolegija Bioinformatika 1 na Fakultetu elektrotehnike i računarstva Sveučilišta u Zagrebu.

\end{document}
